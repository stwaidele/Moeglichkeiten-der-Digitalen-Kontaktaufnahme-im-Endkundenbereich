\section{Grundlagen}
\label{sec:grundlagen}

\subsection{Digitale Kontaktaufnahme}

Definition als Vorgang vorhandene oder potentielle
Kunden vor Ort abzuholen und auf ein digitales bzw. Onlineangebot zu
führen.

\subsection{Notwendigkeit des Medienbruchs}

Erläuterung warum ein Wechsel des Kommunikationsmediums von direkt (bzw. Print) zum Digitalen (Online oder Offline) wünschenswert ist.

\subsection{Eingrenzung der Kundengruppe}

Beschränkung der potentiellen und vorhandenen Kunden kleiner touristischer Betriebe.

\section{Möglichkeiten der digitalen Kontaktaufnahme}

\subsection{QR–Codes}
\subsection{NFC--Tags}
\subsection{Bluetooth}
\subsection{Kurzlinks}
\subsection{Internet--Nachrichtendienste}

z.B. Twitter für nicht an eine Mobilfunknummer gebundene Nachrichtendienste. Auch: iMessage, evt. Facebook Messenger

Aufgrund der hohen Anforderung bzgl. syncroner Kommunikation ist diese Möglichkeit nicht optimal für kleine Betriebe und wird nur im Grundlagenteil behandelt.

\subsection{Mobilfunk--Nachrichtendienste}

z.B. WhatsApp für Nachrichtendienste, die an eine Mobilfunknummer gebunden sind, selbst wenn die eigentliche Kommunikation über Internettechniken abgewickelt wird. Auch: SMS.

Aufgrund der hohen Anforderung bzgl. syncroner Kommunikation ist diese Möglichkeit nicht optimal für kleine Betriebe und wird nur im Grundlagenteil behandelt.

\subsection{Hotel--Apps}

z.B. Protel „Voyager“

Aufgrund der hohen Anforderung bzgl. Kapitaleinsatz ist diese Möglichkeit nicht optimal für kleine Betriebe und wird nur im Grundlagenteil behandelt.

\section{Bewertungskriterien}

\subsection{Benötigte Hard-- und Software (Hotel)}
\subsection{Benötigte Hard-- und Software (Kunde)}
\subsection{Bekanntheitsgrad}
\subsection{Verbreitungsgrad}
\subsection{Benutzerfreundlichkeit}
\subsection{Technische Möglichkeiten}

z.B. Speicherkapazität

\subsection{Mißbrauchsmöglichkeiten}

z.B. überklebte QR--Codes