\section{Grundlagen}
\label{sec:grundlagen}

\subsection{Digitale Kontaktaufnahme}

Definition als Vorgang vorhandene oder potentielle
Kunden vor Ort abzuholen und auf ein digitales bzw. Onlineangebot zu
führen.

\subsection{Gastgewerbe}

\todo{
\url{http://www.dehoga-bundesverband.de/daten-fakten-trends/betriebsarten/}
\\ \url{http://www.dehoga-bundesverband.de/daten-fakten-trends/gastgewerbe-im-ueberblick/}
\\ \url{http://www.dehoga-bundesverband.de/daten-fakten-trends/umsatzentwicklungen/}
\\ \url{http://www.dehoga-bundesverband.de/fileadmin/Inhaltsbilder/Daten_Fakten_Trends/Zahlespiegel_und_Branchenberichte/Zahlenspiegel/Zahlenspiegel_1__Quartal_2014.pdf}
\\ \url{http://www.dehoga-bundesverband.de/presse/pressemitteilungen/gastgewerbe-2013-solides-umsatzplus-von-12-prozent-viertes-jahr-mit-umsatzplus-in-folge-2014-02-17-1000/}
}

\subsection{Eingrenzung der Kundengruppe}

\todo{Beschränkung der potentiellen und vorhandenen Kunden kleiner Betriebe des Gastgewerbes.}

\subsection{Notwendigkeit des Medienbruchs}
\label{sec:medienbruch}
\todo{
Erläuterung warum ein Wechsel des Kommunikationsmediums von direkt (bzw. Print) zum Digitalen (Online oder Offline) wünschenswert ist.

Customer Journey, Reiseanalyse 2014, S5, unten! und http://www.fur.de/ra/news-daten/aktueller-newsletter/nl-0714-die-ra-customer-journey/ 
}
\section{Technologische Möglichkeiten der digitalen Kontaktaufnahme}
\label{sec:technologien}

\subsection{QR–Codes}
\todo{
\url{http://www.nielsen.com/de/de/insights/presseseite/2012/nielsen-das-smartphone-als-shopping-companion.html} \\
\url{http://www.springerprofessional.de/qr-codes-sind-ein-nerd--und-nischengeschichte/4781832.html} \\
\url{http://de.statista.com/statistik/daten/studie/237259/umfrage/bekanntheit-des-begriffes-qr-code-in-deutschland/} }

\subsection{NFC--Tags}
\subsection{Bluetooth}
\subsection{Bluetooth Low Energy}
z.B. iBeacon

\todo{
\url{http://t3n.de/news/apple-ibeacon-nfc-499992/} \\
\url{http://www.huffingtonpost.de/christian-eggert/wie-ibeacon-unser-leben-v_b_5436171.html}\\
\url{http://de.wikipedia.org/wiki/IBeacon} }

\subsection{Kurzlinks}
\subsection{Internet--Nachrichtendienste}

z.B. Twitter für nicht an eine Mobilfunknummer gebundene Nachrichtendienste. Auch: iMessage, evt. Facebook Messenger

Aufgrund der hohen Anforderung bzgl. syncroner Kommunikation ist diese Möglichkeit nicht optimal für kleine Betriebe und wird nur im Grundlagenteil behandelt.

\subsection{Mobilfunk--Nachrichtendienste}

z.B. WhatsApp für Nachrichtendienste, die an eine Mobilfunknummer gebunden sind, selbst wenn die eigentliche Kommunikation über Internettechniken abgewickelt wird. Auch: SMS.

Aufgrund der hohen Anforderung bzgl. syncroner Kommunikation ist diese Möglichkeit nicht optimal für kleine Betriebe und wird nur im Grundlagenteil behandelt.

\subsection{Hotel--Apps}

z.B. Protel „Voyager“

Aufgrund der hohen Anforderung bzgl. Kapitaleinsatz ist diese Möglichkeit nicht optimal für kleine Betriebe und wird nur im Grundlagenteil behandelt.

\section{Bewertungskriterien}
\label{sec:kriterien}

Hier wäre u.U. eine Befragung von Kunden und Betrieben bzgl. der Gewichtung der Kriterien (und zur Gewinnung von weiteren Kriterien) sinnvoll.

\subsection{Benötigte Hard-- und Software (Hotel)}
\subsection{Benötigte Hard-- und Software (Kunde)}
\subsection{Bekanntheitsgrad}
\subsection{Verbreitungsgrad}
\subsection{Benutzerfreundlichkeit}
\subsection{Technische Möglichkeiten}
z.B. Speicherkapazität

\subsection{Finanzielle Anforderungen}
\subsection{Personelle Anforderungen}
\subsection{Mißbrauchsmöglichkeiten}
z.B. überklebte QR--Codes