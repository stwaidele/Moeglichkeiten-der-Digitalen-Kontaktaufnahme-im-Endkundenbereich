\section{Vorstellung der Tourismusbranche}
\label{sec:hauptteil}
\label{sec:tourismusbranche}

Kleine Betriebe, Personalintensiv, Mitarbeiter i.d.R. wenig technikaffin, ...

\section{Bewertung der Technologien}
\label{sec:bewertung}

\subsection{Bewertung QR–Codes}
\subsection{Bewertung NFC--Tags}
\subsection{Bewertung Bluetooth}
\subsection{Bewertung Kurzlinks}

\section{Mögliche Einsatzzwecke}
\label{sec:einsatzzwecke}

\subsection{Kunde mit Informationen versorgen}

VCard, Informationen zum momentanen Ort im Betrieb (Speisesaal--Öffnungszeiten, Bildbeschreibung, ...)

\subsection{Kunde auf eine Webseite führen}
Website des Betriebs, Facebook--Seite, Gewinnspiel, etc.
Auch: Eigene App

\subsection{Bewertungen einsammeln}
Spezialfall der Umleitung auf eine Website

\subsection{Kontaktinformationen Abfragen}
Spezialfall der Umleitung auf eine Website

\subsection{...}

\section{Mögliche Einsatzorte}
\label{sec:einsatzorte}

\subsection{Am POI}
Wo der Kunde die Information haben möchte.

\subsection{Im Lift}
...oder in der Warteschlange, oder wo es sonst dem Kunde langweilig wird.

\subsection{Im Hotelzimmer}
Wo der Gast Zeit hat

\subsection{Auf der Rechnung}
Nach der Reise ist vor der Reise

\subsection{In der Rezeption}

Hier ist eine Situationsbezogene Betrachtung notwendig: Beim Check--In und Check--Out hat der Gast wenig Zeit. Außerdem besteht der direkte Kontakt zu den Mitarbeitern. Hier ist von einem Einsatz der Technologien abzusehen.

Aber: In der Lounge oder Sitzgruppe ist ein Einsatz sinnvoll denkbar.

\subsection{...}

