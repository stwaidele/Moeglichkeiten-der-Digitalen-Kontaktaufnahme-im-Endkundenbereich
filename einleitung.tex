\section{Einleitung}
\label{sec:einleitung}

\subsection{Begründung der Problemstellung}

Im Februar 2014 nutzen 40,4 Millionen Deutsche ein Smartphone\footnote{vgl. \cite{netzoekonom}}. Dies entspricht einem Anteil von knapp 50\% der Bevölkerung\footnote{vgl. \cite{destatis:bev}}, bzw. durschscnittlich eines in fast jedem deutschen Haushalt\footnote{vgl. \cite{destatis:hh}}. 

Mit dieser stetig steigenden Verbreitung von Smartphones in der Bevölkerung nehmen auch die Möglichkeiten der digitalen Kontaktaufnahme zu. 
Hierbei konkurieren verschiedene Technologien miteinander. Allen gemeinsam ist es, dass die Kunden oder Interessenten vor Ort auf digitale, meist Onlineangebote hingewiesen werden.

Unternehmen, die diese Möglichkeiten nutzen möchten, brauchen eine verlässliche Basis für die Entscheidung welche Technologie zum Einsatz kommen sollen. Obwohl technisch nichts dagegen spricht, mehrere Technologien gleichzeitig anzubieten, so nehmen sich diese dann doch den Platz und die Aufmerksamkeit des Kunden.

Wegen des in kleinen Betrieben schon aufgrund der Umsatzhöhe beschränkten Budget sowohl für IT als auch für Marketing ist diese Entscheidung noch bedeutungsvoller als bei großen Konzernen. Konzeption und Umsetzung werden in diesen Betrieben meist von Mitarbeitern erledigt, die gleichzeitig noch andere Aufgaben zu erfüllen haben. Daher sind auch die Bewertungskriterien entsprechend diesen Erfordernissen zu gestalten.

Ein stark wachsender Anteil von 27\% der Reisenden informieren sich wärend ihrer Reise über das mobile Internet. Genze 36\% der Reisendenden  teilen Ihre Reiseerlebnisse im Social Web\footnote{vgl. \cite{reiseanalyse}, Seite 5}. Somit ist das Thema für touristische Betriebe ein Zukunftsthema, das auch schon heute relevant ist.

\subsection{Ziele dieser Arbeit}

Ziel dieses Projekts ist es, Handlungsempfehlungen für den Einsatz von Technologien zur digitalen Kontaktaufnahme in kleinen Betrieben der Tourismusbranche zu geben.

\subsection{Aufbau der Arbeit}

\todo{An die tatsächliche Arbeit anpassen}

Zunächst werden im Kapitel~\ref{sec:grundlagen} die für diese Arbeit relevanten Begriffe und Konzepte definiert, bevor im Kapitel~\ref{sec:technologien} die momentan verfügbaren Technologien genannt und erklärt werden. Die Erarbeitung von Kriterien für die Bewertung und Vergleich der Technologien in Kapitel~\ref{sec:kriterien} wird von zwei Befragungen unterstüzt, die in Anhang~\ref{sec:kundenbefragung} und Anhang~\ref{sec:betriebsbefragung} beschrieben sind. Hiermit schließt der theoretische Teil dieser Arbeit. 

Nach einer kurzen Vorstellung der Tourismusbranche in Kapitel~\ref{sec:tourismusbranche} werden die relevanten Technologien in Kapitel~\ref{sec:bewertung} bewertet und miteinander Verglichen.

In den Kapiteln~\ref{sec:einsatzzwecke} und \ref{sec:einsatzorte} werden dann Handlungsempfehlungen bezüglich möglichen Einsatzzwecken und Einsatzorten gegeben.

\subsection{Abgrenzung}

\todo{An die tatsächliche Arbeit anpassen}

Da der Schritt von der persönlichen  zur unpersönlichen Kommunikation getan wird, sollte untersucht werden, für welche Zielsetzungen dies in Betracht kommt und wünschenswert ist. Da dies den Umfang dieser Arbeit sprengen würde, wird diese Fragestellung nicht ausführlich behandelt.

„Einsatzmöglichkeiten III: Gestaltung“ wird nicht behandelt, hierfür wären spezielle Untersuchungen notwendig.

Usability wird nur anhand von groben Einteilungen behandelt und nicht detailliert untersucht.

Die Technologien Internet--Nachrichtendienste, Mobilfunk--Nachrichtendienste und Hotel––App werden nicht erschöpfend betrachtet werden, da sie aufgrund der Bewertungskriterien Kapital-- bzw. Personaleinsatz offensichtlich nicht optimal für kleine Betriebe sind.