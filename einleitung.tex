\section{Einleitung}
\label{sec:einleitung}


\subsection{Ziele dieser Arbeit}

Mit der zunehmenden Verbreitung von Smartphones bei allen Altergruppen der Bevölkerung nehmen auch die Möglichkeiten der digitalen Kontaktaufnahme zu. Hierbei konkurieren verschiedene Technologien miteinander. Unternehmen, die diese Möglichkeiten nutzen möchten, brauchen eine verlässliche Basis für die Entscheidung welche Technologie bzw. Technilogien zum Einsatz kommen sollen. Aufgrund des in kleinen Betrieben oft beschränkten Budget sowohl für IT als auch für Marketing ist diese Entscheidung noch bedeutungsvoller als bei großen Konzernen.

Im Rahmen dieses Projekts sollen die Möglichkeiten benannt und untersucht werden, mit denen Betriebe der Tourismusbranche mit Kunden vor Ort digital in Kontakt treten können. 

Da hierbei wird der Schritt von der persönlichen  zur unpersönlichen Kommunikation getan wird, soll weiterhin untersucht werden, für welche Zielsetzungen dies in Betracht kommt und wünschenswert ist.

Die genannten Technologien sollen anhand von Bewertungskriterien beurteilt und verglichen werden.

Abschließend sollen Handlungsempfehlungen für den Einsatz in kleinen Betrieben der Tourismusbranche gegeben werden.


\subsection{Aufbau der Arbeit}

Zunächst werden im Kapitel~\ref{sec:grundlagen} die für diese Arbeit relevanten Begriffe und Konzepte definiert, bevor im Kapitel~\ref{sec:technologien} die momentan verfügbaren Technologien genannt und erklärt werden. Mit der Erarbeitung von Kriterien für die Bewertung und den Vergleich dieser Technologien in Kapitel~\ref{sec:kriterien} schließt der theoretische Teil dieser Arbeit 

Nach einer kurzen Vorstellung der Tourismusbranche in Kapitel~\ref{sec:tourismusbranche} werden die relevanten Technologien in Kapitel~\ref{sec:bewertung} bewertet und miteinander Verglichen.

In den Kapiteln~\ref{sec:einsatzzwecke} und \ref{sec:einsatzorte} werden dann Handlungsempfehlungen bezüglich möglichen Einsatzzwecken und Einsatzorten gegeben.

\subsection{Abgrenzung}

„Einsatzmöglichkeiten III: Gestaltung“ wird nicht behandelt, hierfür wären spezielle Untersuchungen notwendig.

Usability wird nur anhand von groben Einteilungen behandelt und nicht detailliert untersucht.

Die Technologien Internet--Nachrichtendienste, Mobilfunk--Nachrichtendienste und Hotel––App werden nicht erschöpfend betrachtet werden, da sie aufgrund der Bewertungskriterien Kapital-- bzw. Personaleinsatz offensichtlich nicht optimal für kleine Betriebe sind.